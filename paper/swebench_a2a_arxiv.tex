\documentclass[11pt,twocolumn]{article}
\usepackage[margin=0.75in]{geometry}
\usepackage{times}
\usepackage{hyperref}
\usepackage{graphicx}
\usepackage{amsmath, amssymb}
\usepackage{enumitem}
\usepackage{booktabs}
\usepackage{xcolor}
\usepackage{listings}
\usepackage{algorithm}
\usepackage{algorithmic}

\definecolor{codegreen}{rgb}{0,0.6,0}
\definecolor{codegray}{rgb}{0.5,0.5,0.5}
\definecolor{codepurple}{rgb}{0.58,0,0.82}

\lstset{
    basicstyle=\ttfamily\footnotesize,
    keywordstyle=\color{codepurple},
    commentstyle=\color{codegreen},
    stringstyle=\color{codegray},
    breaklines=true,
    frame=single
}

\title{\textbf{SWE-Bench-A2A: Process-Aware, Contamination-Resistant\\Evaluation of Software Engineering Agents}\\
\large{via Agent-to-Agent Protocol}}

\author{
Manish Muttreja\\
\texttt{manish@example.com}
}

\date{}

\begin{document}
\maketitle

\begin{abstract}
SWE-bench has emerged as the de facto standard for evaluating language model agents on real-world software engineering tasks, using GitHub issues and execution-based testing. However, the benchmark suffers from three critical limitations: (1) \textbf{data contamination}---models may have memorized repositories and patches during pretraining; (2) \textbf{patch-only scoring}---evaluation ignores the engineering process, rewarding lucky guesses equally with systematic debugging; and (3) \textbf{static test dependence}---fixed test suites can be overfit without true understanding. We present \textbf{SWE-Bench-A2A}, an extension that addresses these limitations through four key innovations: a \textit{reproduction-first gate} requiring agents to demonstrate bug understanding before patching, \textit{trajectory-based process scoring} capturing the full engineering workflow, \textit{anti-memorization mutations} via retro-holdout transformations, and \textit{dynamic adversarial testing} beyond static suites. Our implementation uses an Agent-to-Agent (A2A) protocol where a Green Agent (assessor) orchestrates evaluation of Purple Agents (solvers) in isolated Docker environments. We demonstrate end-to-end functionality on a 3-instance SWE-bench Verified sample: one instance passed using a heuristic fallback patch, while two failed due to invalid LLM-generated patches from GPT-4o-mini, highlighting the need for stronger solver models. We release the framework as Docker images for integration with the AgentBeats evaluation platform.
\end{abstract}

\section{Introduction}

The rapid advancement of large language models (LLMs) has enabled a new class of \textit{software engineering agents}---systems that can understand codebases, diagnose bugs, and generate patches with minimal human intervention. Evaluating these agents requires benchmarks that capture the complexity of real-world software engineering while resisting the pitfalls of static evaluation.

SWE-bench \cite{jimenez2024swebench} represents a significant step forward, drawing from 2,294 real GitHub issues across 12 popular Python repositories. Unlike synthetic benchmarks, SWE-bench tasks require agents to navigate complex codebases, understand issue descriptions, and produce patches that pass repository test suites. This execution-based evaluation provides a strong signal of functional correctness.

However, as SWE-bench has become ubiquitous in agent evaluation, three fundamental limitations have emerged:

\begin{enumerate}[leftmargin=*]
    \item \textbf{Data Contamination}: The repositories in SWE-bench (Django, Flask, Scikit-learn, etc.) are among the most common in LLM training corpora. Models may have memorized not just the codebases but the specific patches that resolve benchmark issues.
    
    \item \textbf{Patch-Only Scoring}: Current evaluation awards full credit for any patch that passes tests, ignoring whether the agent understood the problem. A model that guesses correctly receives the same score as one that systematically debugged the issue.
    
    \item \textbf{Static Test Dependence}: Fixed test suites can be overfit through pattern matching without true understanding. Agents may learn to produce patches that pass specific tests while failing on equivalent formulations.
\end{enumerate}

We present \textbf{SWE-Bench-A2A}, an evaluation framework that addresses these limitations through four key innovations:

\begin{itemize}[leftmargin=*]
    \item \textbf{Reproduction Gate}: Agents must first produce a failing test that reproduces the bug, demonstrating understanding before patching.
    
    \item \textbf{Process Scoring}: Beyond pass/fail, we capture full agent trajectories and compute multi-dimensional scores for correctness, process quality, efficiency, and adaptation.
    
    \item \textbf{Anti-Memorization}: Retro-holdout mutations transform codebases with semantic-preserving renames, and a fresh issue harvester provides never-before-seen tasks.
    
    \item \textbf{Dynamic Testing}: Beyond repository tests, we support fuzz testing, mutation testing, and adversarial probes to detect overfitting.
\end{itemize}

Our framework implements the Agent-to-Agent (A2A) protocol, enabling modular composition of assessors (Green Agents) and participants (Purple Agents). This design allows any solver to be evaluated without modification, promoting reproducibility and fair comparison.

\section{Related Work}

\subsection{Code Generation Benchmarks}

Early code benchmarks like HumanEval \cite{chen2021codex} and MBPP \cite{austin2021mbpp} evaluate function-level generation from docstrings. While useful for measuring basic coding ability, these synthetic tasks lack the complexity of real software engineering: multi-file reasoning, dependency management, and test integration.

\subsection{Repository-Level Evaluation}

SWE-bench \cite{jimenez2024swebench} pioneered repository-level evaluation using real GitHub issues. The SWE-bench Verified subset provides human-validated instances with clearer specifications. Concurrent work like DevBench \cite{devbench2024} extends to multi-language settings.

\subsection{Contamination and Memorization}

Data contamination in LLM benchmarks has been extensively documented \cite{contamination2023}. For code benchmarks, the problem is acute: popular repositories appear repeatedly in training data. Techniques like canary strings and holdout sets provide partial mitigation but cannot detect memorization of existing public data.

\subsection{Process-Aware Evaluation}

Traditional software engineering emphasizes process quality alongside outcomes. Test-driven development (TDD) requires understanding before implementation. Our reproduction gate operationalizes this principle for agent evaluation.

\section{Limitations of Current SWE-bench}

\subsection{Data Contamination}

SWE-bench repositories are among the most-starred Python projects on GitHub. Analysis suggests substantial overlap with common training corpora:

\begin{itemize}[leftmargin=*]
    \item Django: 76k+ stars, extensive documentation
    \item Flask: 66k+ stars, widely referenced in tutorials
    \item Scikit-learn: 58k+ stars, standard ML library
\end{itemize}

Models trained on web-scale data have likely seen these codebases, their issues, and their patches. Performance on ``unseen'' tasks may reflect recall rather than reasoning.

\subsection{Patch-Only Evaluation}

Current scoring treats all passing patches equally:
\begin{equation}
    \text{Score} = \mathbb{1}[\text{all tests pass}]
\end{equation}

This binary metric ignores:
\begin{itemize}[leftmargin=*]
    \item Whether the agent understood the bug
    \item The quality of the debugging process
    \item Efficiency of the solution path
    \item Ability to handle ambiguity
\end{itemize}

\subsection{Static Test Overfitting}

Repository test suites, while valuable, have fixed specifications. Agents may learn patterns that satisfy specific tests without generalizing. A patch that passes \texttt{test\_user\_login} may fail on semantically equivalent \texttt{test\_account\_authentication}.

\section{SWE-Bench-A2A Design}

\subsection{A2A Protocol Architecture}

Our framework implements the Agent-to-Agent protocol with two actor types:

\begin{description}
    \item[Green Agent (Assessor)] Orchestrates evaluation: provisions environments, dispatches tasks, verifies solutions, computes scores.
    
    \item[Purple Agent (Solver)] Attempts tasks: receives issue descriptions, explores codebases, generates patches.
\end{description}

Communication occurs via REST endpoints with standardized message formats:

\begin{lstlisting}[language=Python]
# Task creation
POST /a2a/task
{
  "title": "Fix bug #1234",
  "description": "...",
  "resources": {"repo": "...", "commit": "..."}
}

# Artifact submission  
POST /a2a/task/{id}/artifact
{
  "type": "patch_submission",
  "parts": [{"type": "file_diff", "content": "..."}]
}
\end{lstlisting}

This separation enables any solver to be evaluated without code changes, promoting fair comparison across systems.

\subsection{Reproduction Gate}

Before accepting patches, we require agents to demonstrate bug understanding through reproduction:

\begin{algorithm}
\caption{Reproduction Gate Protocol}
\begin{algorithmic}[1]
\REQUIRE Issue description $I$, environment $E$
\STATE Agent submits reproduction script $R$
\STATE Execute $R$ in unpatched $E$
\IF{$R$ does not fail}
    \STATE \textbf{reject}: ``Reproduction must fail before patch''
\ENDIF
\STATE Agent submits patch $P$
\STATE Apply $P$ to $E$
\STATE Execute $R$ in patched $E$
\IF{$R$ still fails}
    \STATE \textbf{reject}: ``Patch does not fix reproduction''
\ENDIF
\STATE Run full test suite
\RETURN verification result
\end{algorithmic}
\end{algorithm}

This gate enforces test-driven development principles: understand the problem (red), then fix it (green).

\subsection{Trajectory-Based Process Scoring}

We capture complete agent trajectories and compute multi-dimensional scores:

\begin{equation}
    S = \sum_{c \in \mathcal{C}} w_c \cdot s_c
\end{equation}

where $\mathcal{C}$ includes:

\begin{table}[h]
\centering
\small
\begin{tabular}{lrl}
\toprule
Category & Weight & Description \\
\midrule
Correctness & 0.35 & Tests pass, patch applies \\
Process & 0.20 & Systematic exploration \\
Efficiency & 0.15 & Token/time usage \\
Collaboration & 0.15 & Information requests \\
Understanding & 0.10 & Reproduction quality \\
Adaptation & 0.05 & Response to feedback \\
\bottomrule
\end{tabular}
\caption{Scoring dimensions and weights}
\end{table}

\subsection{Anti-Memorization Strategies}

\subsubsection{Retro-Holdout Mutations}

We transform codebases with semantic-preserving mutations:

\begin{itemize}[leftmargin=*]
    \item \textbf{Variable renaming}: \texttt{data} $\to$ \texttt{payload}
    \item \textbf{Function renaming}: \texttt{get\_user} $\to$ \texttt{fetch\_account}
    \item \textbf{Class renaming}: \texttt{UserManager} $\to$ \texttt{AccountHandler}
    \item \textbf{Comment perturbation}: Rephrase docstrings
\end{itemize}

Mutations are applied consistently across the codebase while preserving test behavior. This creates ``parallel universes'' where memorized patches no longer apply.

\subsubsection{Fresh Issue Harvesting}

A harvester monitors GitHub for new issues in target repositories, providing tasks created after model training cutoffs. These ``secret-in-time'' instances provide contamination-free evaluation.

\subsection{Dynamic Testing}

Beyond repository tests, we support:

\begin{description}
    \item[Fuzz Testing] Property-based tests with random inputs
    \item[Mutation Testing] Assert patches handle code mutations
    \item[Adversarial Probes] LLM-generated edge cases
\end{description}

\section{Implementation}

\subsection{System Architecture}

The implementation consists of several key components:

\begin{itemize}[leftmargin=*]
    \item \textbf{A2A Server}: FastAPI-based REST API implementing the A2A protocol with endpoints for task management, artifact submission, and health checks.
    
    \item \textbf{Environment Orchestrator}: Docker-based container management with JIT provisioning, repository cloning, and commit checkout.
    
    \item \textbf{Verification Engine}: Patch application, test execution with timeout handling, and flaky test detection.
    
    \item \textbf{Trajectory Capture}: Action logging with database persistence and streaming support.
    
    \item \textbf{LLM Solver}: Integration with OpenAI/Anthropic APIs for reproduction script and patch generation. The solver includes a three-tier fallback hierarchy: (1) real LLM API calls when API keys are configured, (2) heuristic patches for known benchmark instances (e.g., django-11099), and (3) mock responses when no API access is available. This design enables both production evaluation with frontier models and development testing without API costs.
\end{itemize}

\subsection{Docker Images}

We provide containerized agents for easy deployment:

\begin{lstlisting}[language=bash]
# Green Agent (Assessor)
docker.io/mmuttreja761/swebench-a2a-green:1.0.0

# Purple Agent (Solver)
docker.io/mmuttreja761/swebench-a2a-purple:1.0.0
\end{lstlisting}

These images are compatible with the AgentBeats evaluation platform for standardized benchmarking.

\section{Experiments}

\subsection{Setup}

We evaluated the framework on a 3-instance sample from SWE-bench Verified:

\begin{itemize}[leftmargin=*]
    \item \texttt{django\_\_django-11099}: UsernameValidator trailing newline
    \item \texttt{django\_\_django-11133}: HttpResponse charset handling
    \item \texttt{django\_\_django-11179}: model\_to\_dict for unsaved model
\end{itemize}

The Purple Agent was configured with GPT-4o-mini for LLM calls, with heuristic fallbacks enabled for known instances.

\subsection{Results}

\begin{table}[h]
\centering
\small
\begin{tabular}{lcccc}
\toprule
Instance & Patch & Tests & Time & Source \\
\midrule
django-11099 & \checkmark & 3/3 & 74s & Heuristic \\
django-11133 & $\times$ & 0/0 & 73s & LLM \\
django-11179 & $\times$ & 0/0 & 71s & LLM \\
\midrule
\textbf{Total} & 33.3\% & - & - & - \\
\bottomrule
\end{tabular}
\caption{Evaluation results on 3-instance sample. ``Source'' indicates whether the patch came from a heuristic fallback or LLM generation.}
\end{table}

The successful case (django-11099) demonstrated the full end-to-end pipeline:
\begin{enumerate}[leftmargin=*]
    \item Environment provisioned with correct commit (d26b2424)
    \item Task dispatched to Purple Agent via A2A protocol
    \item Patch applied (from heuristic fallback) and verified
    \item All 3 oracle tests passed
\end{enumerate}

\textbf{Key observation}: The LLM-generated patches (django-11133, django-11179) failed to apply due to malformed unified diff format. This highlights a critical gap: while the \textit{framework infrastructure} (Docker provisioning, A2A communication, test execution) is fully operational, the \textit{solver quality} depends heavily on the underlying LLM's ability to produce syntactically valid patches. Stronger models (GPT-4, Claude 3.5 Sonnet) or improved prompting strategies are needed for production use.

\subsection{Trajectory Analysis}

For the successful case, the captured trajectory shows:

\begin{lstlisting}
1. scenario_select -> django__django-11099
2. provision_environment -> [container_id]
3. dispatch_task -> [purple_task_id]
4. receive_artifact -> reproduction_script
5. receive_artifact -> patch_submission
6. verification -> passed (3/3 tests)
\end{lstlisting}

This visibility enables debugging agent behavior and computing process scores.

\section{Evaluation Slices}

We propose four evaluation slices for comprehensive assessment:

\begin{description}
    \item[Verified] Standard SWE-bench Verified instances
    \item[Mutated] Retro-holdout transformed versions
    \item[Fresh] Newly harvested issues (<24h old)
    \item[Adversarial] Instances with fuzz/mutation testing
\end{description}

Reporting across slices reveals contamination sensitivity and robustness.

\section{Limitations and Future Work}

\subsection{Current Limitations}

\begin{itemize}[leftmargin=*]
    \item \textbf{Python only}: Current implementation focuses on Python repositories
    \item \textbf{LLM solver quality}: GPT-4o-mini produces invalid unified diff patches for most issues. Our experiments show 0\% success on novel instances without heuristic fallbacks. Production deployment requires either (a) stronger models like GPT-4 or Claude 3.5 Sonnet, (b) few-shot prompting with diff format examples, or (c) multi-turn correction loops.
    \item \textbf{Heuristic fallbacks}: The current implementation includes hardcoded patches for known SWE-bench instances (e.g., django-11099). While useful for demonstrating framework functionality, these should be removed for fair benchmarking.
    \item \textbf{Mutation coverage}: Retro-holdout not yet integrated in live evaluation flow
    \item \textbf{Dynamic test generation}: Fuzz/adversarial commands require per-repo configuration
\end{itemize}

\subsection{Future Directions}

\begin{enumerate}[leftmargin=*]
    \item Integrate stronger models (Claude 3.5, GPT-4) for Purple agent
    \item Complete retro-holdout pipeline with semantic equivalence verification
    \item Implement default fuzz command packs for common frameworks
    \item Extend to multi-language evaluation (TypeScript, Rust)
    \item Add visual/multimodal signals for UI-related bugs
\end{enumerate}

\section{Conclusion}

SWE-Bench-A2A addresses critical limitations in current software engineering agent evaluation. By requiring reproduction before patching, capturing process quality through trajectories, applying anti-memorization transformations, and supporting dynamic testing, we provide a more rigorous and informative benchmark.

Our A2A protocol design enables modular composition of assessors and solvers, promoting reproducibility and fair comparison. The framework is available as Docker images compatible with the AgentBeats evaluation platform.

As language model agents become increasingly capable at software engineering, robust evaluation becomes critical. SWE-Bench-A2A offers a path toward contamination-resistant, process-aware benchmarking that rewards true engineering ability over memorization.

\section*{Acknowledgments}

We thank the SWE-bench team for their foundational work and the AgentBeats community for evaluation infrastructure.

\bibliographystyle{plain}
\begin{thebibliography}{9}

\bibitem{jimenez2024swebench}
C.~E. Jimenez, J.~Yang, A.~Wettig, S.~Yao, K.~Pei, O.~Press, and K.~Narasimhan.
\newblock SWE-bench: Can language models resolve real-world GitHub issues?
\newblock In \textit{International Conference on Learning Representations (ICLR)}, 2024.

\bibitem{chen2021codex}
M.~Chen, J.~Tworek, H.~Jun, Q.~Yuan, H.~Pinto, J.~Kaplan, et al.
\newblock Evaluating large language models trained on code.
\newblock \textit{arXiv preprint arXiv:2107.03374}, 2021.

\bibitem{austin2021mbpp}
J.~Austin, A.~Odena, M.~Nye, M.~Bosma, H.~Michalewski, D.~Dohan, et al.
\newblock Program synthesis with large language models.
\newblock \textit{arXiv preprint arXiv:2108.07732}, 2021.

\bibitem{devbench2024}
DevBench Team.
\newblock DevBench: A comprehensive benchmark for software development.
\newblock \textit{arXiv preprint}, 2024.

\bibitem{contamination2023}
O.~Sainz, J.~Campos, I.~García-Ferrero, J.~Etxaniz, O.~Lopez de Lacalle, and E.~Agirre.
\newblock NLP evaluation in trouble: On the need to measure LLM data contamination for each benchmark.
\newblock In \textit{Findings of EMNLP}, 2023.

\bibitem{agentbeats}
AgentBeats Platform.
\newblock Agent registry for AI evaluation.
\newblock \url{https://agentbeats.dev/}, 2024.

\end{thebibliography}

\appendix

\section{A2A Protocol Specification}

\subsection{Agent Card Format}

\begin{lstlisting}[language=json]
{
  "name": "SWE-bench Green Agent",
  "version": "1.0.0",
  "agent_id": "uuid",
  "capabilities": ["swebench_evaluation"],
  "endpoints": {
    "task": "/a2a/task",
    "health": "/health"
  }
}
\end{lstlisting}

\subsection{Artifact Types}

\begin{itemize}
    \item \texttt{reproduction\_script}: CODE artifact with failing test
    \item \texttt{patch\_submission}: FILE\_DIFF artifact with unified diff
    \item \texttt{assessment\_result}: JSON artifact with verification results
\end{itemize}

\section{Scoring Formula Details}

\subsection{Correctness Score}
\begin{equation}
    s_{\text{correct}} = 0.6 \cdot \mathbb{1}[\text{pass}] + 0.3 \cdot \frac{\text{tests\_passed}}{\text{total\_tests}} + 0.1 \cdot \mathbb{1}[\text{patch\_applied}]
\end{equation}

\subsection{Process Score}
\begin{equation}
    s_{\text{process}} = 0.4 \cdot s_{\text{exploration}} + 0.3 \cdot s_{\text{reasoning}} + 0.3 \cdot s_{\text{reproduction}}
\end{equation}

\subsection{Efficiency Score}
\begin{equation}
    s_{\text{efficiency}} = 0.4 \cdot \frac{T_{\text{budget}} - T_{\text{used}}}{T_{\text{budget}}} + 0.4 \cdot \frac{N_{\text{budget}} - N_{\text{tokens}}}{N_{\text{budget}}} + 0.2 \cdot \frac{1}{1 + \text{attempts}}
\end{equation}

\end{document}
